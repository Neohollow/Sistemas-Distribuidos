%%\documentclass[a4paper,12pt,oneside]{llncs}
\documentclass[12pt,letterpaper]{article}
\usepackage[right=2cm,left=3cm,top=2cm,bottom=2cm,headsep=0cm]{geometry}

%%%%%%%%%%%%%%%%%%%%%%%%%%%%%%%%%%%%%%%%%%%%%%%%%%%%%%%%%%%
%% Juego de caracteres usado en el archivo fuente: UTF-8
\usepackage{ucs}
\usepackage[utf8x]{inputenc}

%%%%%%%%%%%%%%%%%%%%%%%%%%%%%%%%%%%%%%%%%%%%%%%%%%%%%%%%%%%
%% Juego de caracteres usado en la salida dvi
%% Otra posibilidad: \usepackage{t1enc}
\usepackage[T1]{fontenc}

%%%%%%%%%%%%%%%%%%%%%%%%%%%%%%%%%%%%%%%%%%%%%%%%%%%%%%%%%%%
%% Ajusta maergenes para a4
%\usepackage{a4wide}

%%%%%%%%%%%%%%%%%%%%%%%%%%%%%%%%%%%%%%%%%%%%%%%%%%%%%%%%%%%
%% Uso fuente postscript times, para que los ps y pdf queden y pequeños...
\usepackage{times}

%%%%%%%%%%%%%%%%%%%%%%%%%%%%%%%%%%%%%%%%%%%%%%%%%%%%%%%%%%%
%% Posibilidad de hipertexto (especialmente en pdf)
%\usepackage{hyperref}
\usepackage[bookmarks = true, colorlinks=true, linkcolor = black, citecolor = black, menucolor = black, urlcolor = black]{hyperref}

%%%%%%%%%%%%%%%%%%%%%%%%%%%%%%%%%%%%%%%%%%%%%%%%%%%%%%%%%%%
%% Graficos 
\usepackage{graphics,graphicx}

%%%%%%%%%%%%%%%%%%%%%%%%%%%%%%%%%%%%%%%%%%%%%%%%%%%%%%%%%%%
%% Ciertos caracteres "raros"...
\usepackage{latexsym}

%%%%%%%%%%%%%%%%%%%%%%%%%%%%%%%%%%%%%%%%%%%%%%%%%%%%%%%%%%%
%% Matematicas aun más fuertes (american math dociety)
\usepackage{amsmath}

%%%%%%%%%%%%%%%%%%%%%%%%%%%%%%%%%%%%%%%%%%%%%%%%%%%%%%%%%%%
\usepackage{multirow} % para las tablas
\usepackage[spanish,es-tabla]{babel}

%%%%%%%%%%%%%%%%%%%%%%%%%%%%%%%%%%%%%%%%%%%%%%%%%%%%%%%%%%%
%% Fuentes matematicas lo mas compatibles posibles con postscript (times)
%% (Esto no funciona para todos los simbolos pero reduce mucho el tamaño del
%% pdf si hay muchas matamaticas....
\usepackage{mathptm}

%%% VARIOS:
\usepackage{slashbox}
\usepackage{verbatim}
\usepackage{array}
\usepackage{listings}
\usepackage{multirow}

%% MARCA DE AGUA
%% Este package de "draft copy" NO funciona con pdflatex
%%\usepackage{draftcopy}
%% Este package de "draft copy" SI funciona con pdflatex
%%%\usepackage{pdfdraftcopy}
%%%%%%%%%%%%%%%%%%%%%%%%%%%%%%%%%%%%%%%%%%%%%%%%%%%%%%%%%%%
%% Indenteacion en español...
\usepackage[spanish]{babel}

\usepackage{listings}
% Para escribir código en C
% \begin{lstlisting}[language=C]
% #include <stdio.h>
% int main(int argc, char* argv[]) {
% puts("Hola mundo!");
% }
% \end{lstlisting}


\title{Grado en Ingeniería Informática\\Sistemas Distribuidos\\Práctica 3}
\author{José Manuel Morales García \\ José Joaquín Pérez-Calderón Ortiz}

\begin{document}
	\maketitle
	\begin{center}
		\includegraphics[scale=0.22]{C:/Users/trico/Desktop/sd/P3/Imagenes/uca}
	\end{center}
	
	\thispagestyle{empty}
	\newpage
	\tableofcontents
	\newpage
	
	%%\listoftables
	%%\newpage
	
	%%\listoffigures
	%%\newpage
	
	%%%% REAL WORK BEGINS HERE:
	
	%%Configuracion del paquete listings
	\lstset{language=bash, numbers=left, numberstyle=\tiny, numbersep=10pt, firstnumber=1, stepnumber=1}
		\section{Tecnologías empleadas.}
			Estamos empleando Python 3
			Nosotros hemos decidido emplear las siguientes tecnologías.(Explicación de dado que nuestro proyecto bla bla bla)
			\subsection{Python}
			\subsection{Twitter}
			\subsection{Drive}
			\subsection{Rabbitmq }
			\subsection{SMPT}
		\newpage	
		\section{Descripción del flujo y funcionamiento}
			\subsection{Recolección de datos}
			\subsection{Tratado de los datos}
			\subsection{Subida a Drive}
			\subsection{Comparación de valores e impresión}
			\subsection{Dibujo de gráfica mediante ...}
		\newpage	
		\section{Demostración de funcionamiento}
		\newpage
		\section{Referencias}
			\begin{itemize}
				\item Python 3\\→ https://docs.python.org/3/tutorial/
				\item Beautifulsoup:\\→ https://www.crummy.com/software/BeautifulSoup/bs4/doc/\\
				\item Requests: \\→ http://docs.python-requests.org/en/master/
				\item 4\\→
				\item 5\\→
				\item 6\\→
				\item 7\\→
			\end{itemize}
		\newpage
		\section{Valoraciones}
			\begin{table}[htp]
				\resizebox{17cm}{!}{
					\begin{tabular}{|l|c|c|c|c|}
						\hline
						\textbf{Concepto}                                                        & \multicolumn{1}{l|}{\textbf{Aprobado}}                                    & \multicolumn{1}{l|}{\textbf{Notable}}                                                      & \multicolumn{1}{l|}{\textbf{Sobresaliente}}                              & \textbf{Porcentaje \%} \\ \hline
						\textbf{Tecnologías usadas}                                              & Dos                                                                       & Tres                                                                                       & Cuatro o más                                                             & 30\%                   \\ \hline
						\textbf{Almacenamiento}                                                  & No controla errores                                                       & Algún control o no asíncrono                                                               & \begin{tabular}[c]{@{}c@{}}Control de errores y\\ asíncrono\end{tabular} & 10\%                   \\ \hline
						\textbf{\begin{tabular}[c]{@{}l@{}}Peticiones\\ asíncronas\end{tabular}} & Alguna                                                                    & Varios con algún error                                                                     & Varias y sin errores                                                     & 10\%                   \\ \hline
						\textbf{\begin{tabular}[c]{@{}l@{}}Procesamiento\\ salida\end{tabular}}  & Apenas procesamiento                                                      & Procesamiento medio                                                                        & Procesamiento interesante                                                & 10\%                   \\ \hline
						\textbf{Dificultad}                                                      & Demasiado fácil                                                           & Complejidad algo baja                                                                      & Suficientemente complejo                                                 & 15\%                   \\ \hline
						\textbf{Escalabilidad}                                                   & \begin{tabular}[c]{@{}c@{}}Escala muy básica o con\\ errores\end{tabular} & \begin{tabular}[c]{@{}c@{}}Escala con limitaciones\\ (no en lo importante)\end{tabular}    & Escala bien                                                              & 15\%                   \\ \hline
						\textbf{Documentación}                                                   & Falta funcionalidad, dudas                                                & \begin{tabular}[c]{@{}c@{}}Bien la funcionalidad\\ mejorable la documentación\end{tabular} & Bien ambos conceptos                                                     & 10\%                   \\ \hline
					\end{tabular}
				}
			\end{table}
\end{document}